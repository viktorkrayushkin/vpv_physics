\section{Theoretical background}
\begin{frame}{Первые уравнения,начало развития}
\begin{block}{Модель}
Прогнозирование температуры поверхности стенки в начале стабильного плёночного кипения.
\end{block}
\vspace{0.10cm}
 У модели две особенности:
  \vspace{0.05cm}
 \begin{itemize}
\item первое, то что температура стенки, при которой начинается стабильное плёночное кипение - это foam limit (предел пены), то есть максимальная температура, до которой жидкость может быть перегрета.
 \vspace{0.05cm}
\item второе, то что предел пенообразования можно вычислить и довольно точно из уравнения Ван-дер-Ваальса. Экспериментальные  данные о начале стабильного кипения плёнки хорошо согласуются с температурами стенки, предсказанными этой моделью. 
\end{itemize}
\end{frame}